%!TEX TS-program = xelatex

\documentclass[12pt,a4paper]{article}

\RequirePackage[hmargin=2.5cm,vmargin=2.5cm]{geometry}

\usepackage[T1]{fontenc}
\usepackage{lmodern}
\usepackage{amssymb,amsmath}
\usepackage{tcolorbox}
\usepackage{textcomp}
\usepackage{ifxetex,ifluatex}
\usepackage{fixltx2e} % provides \textsubscript
\usepackage{mathptmx}
\usepackage{caption}

\usepackage{lineno}


\usepackage{csvsimple}

\DeclareCaptionFont{xipt}{\fontsize{12}{14}\mdseries}
\usepackage[margin=10pt,font=xipt,labelfont=bf,justification=raggedleft]{caption}
\usepackage{wrapfig}

\usepackage{longtable}
\usepackage{booktabs}

%ns
\usepackage{colortbl}
\usepackage{pifont}
%\documentclass[UTF8]{ctexart}
\usepackage{amsmath}
\usepackage{amssymb}
\usepackage{graphicx}
\usepackage{epstopdf}
\usepackage{inputenc}
%

\usepackage{fontspec}
\usepackage{xcolor}

\usepackage{titlesec}
\defaultfontfeatures{Ligatures=TeX}
\setsansfont{Times New Roman}
\setmainfont{Times New Roman}

\pagenumbering{gobble}

\usepackage{array}
\newcolumntype{L}[1]{>{\raggedright\let\newline\\\arraybackslash\hspace{0pt}}m{#1}}
\newcolumntype{C}[1]{>{\centering\let\newline\\\arraybackslash\hspace{0pt}}m{#1}}
\newcolumntype{R}[1]{>{\raggedleft\let\newline\\\arraybackslash\hspace{0pt}}m{#1}}



% Set formats for each heading level
\titleformat*{\section}{\fontsize{12}{14}\bfseries\sffamily}
\titleformat*{\subsection}{\fontsize{12}{14}\bfseries\sffamily}
\titleformat*{\subsubsection}{\itshape\fontsize{12}{14}\sffamily}

% set title space
%\titlespacing{command}{left spacing}{before spacing}{after spacing}[right]
% spacing: how to read {12pt plus 4pt minus 2pt}
%           12pt is what we would like the spacing to be
%           plus 4pt means that TeX can stretch it by at most 4pt
%           minus 2pt means that TeX can shrink it by at most 2pt
%       This is one example of the concept of, 'glue', in TeX

\titlespacing\section{0pt}{10pt plus 4pt minus 2pt}{6pt plus 2pt minus 2pt}
\titlespacing\subsection{0pt}{10pt plus 4pt minus 2pt}{0pt plus 2pt minus 2pt}
\titlespacing\subsubsection{0pt}{10pt plus 4pt minus 2pt}{0pt plus 2pt minus 2pt}



% Reduce space in lists
\usepackage{enumitem}
\setlist{nosep} % or \setlist{noitemsep} to leave space around whole list





\usepackage{caption}
\captionsetup[figure]{labelfont={bf}, justification=raggedright, singlelinecheck=false, font={stretch=1.5}}
\usepackage[labelsep=none]{caption}

% fix floating figures
\usepackage{float}
\let\origfigure\figure
\let\endorigfigure\endfigure
\renewenvironment{figure}[1][2] {
    \expandafter\origfigure\expandafter[H]
} {
    \endorigfigure
}


\ifxetex
  \usepackage[setpagesize=false, % page size defined by xetex
              unicode=false, % unicode breaks when used with xetex
              xetex]{hyperref}
\else
  \usepackage[unicode=true]{hyperref}
\fi
\hypersetup{breaklinks=true,
            bookmarks=true,
            pdfauthor={},
            pdftitle={},
            colorlinks=true,
            urlcolor=black,
            linkcolor=black,
            pdfborder={0 0 0}}

\usepackage{nameref}

\usepackage{zref-xr}

\zxrsetup{tozreflabel=false, toltxlabel=true, verbose}
\zexternaldocument*{equations_MS_SI}



\usepackage{setspace}
\doublespacing

\setlength{\parindent}{15pt}
\setlength{\parskip}{6pt plus 2pt minus 1pt}
\setlength{\emergencystretch}{3em}  % prevent overfull lines

\def\tightlist{}

\setcounter{secnumdepth}{0}

\usepackage[document]{ragged2e}

\definecolor{lightgrey}{HTML}{eeeeee}

% NMDS add line numbers
\usepackage{lineno}
\linenumbers

\begin{document}


\usepackage[dvipsnames]{xcolor}

\hypertarget{supplementary-methods}{%
\section{Supplementary methods}\label{supplementary-methods}}

To apply the bioenergetic model that estimates fluxes of carbon (C),
nitrogen (N), and phosphorus (P), a number of parameters are required
(Table 1). Here, we describe how these parameters were quantified for
all 1110 species in our database, with a comvbination of literature,
empirical measures, and Bayesian models. All analysis wes carried out in
R v.3.6.3 (REF) and Bayesian modes were run using Stan (REF) and the R
package brms (REF).

\begin{table}[h!]
\centering
\caption{. Overview of model input parameters. VBGC = von Bertalanffy growth curve.}
\begin{tabular}{l l l}
\hline
\rowcolor{lightgrey}
Symbol & Description & Unit\\
\hline
$k$ & Index for element C, N or P & \_ \\
$a_\textrm{k}$  & Element-specific assimilation efficiency & \% \\
$l_\textrm{t}$  & Total length of individual at time t & cm \\
$l_{\infty}$  & Asymptotic adult length (VBGC) & cm \\
$\kappa$  & Growth rate parameter (VBGC) & $\textrm{yr}^{-1}$ \\
$t_0$  & Age at settlement (VBGC) & $\textrm{yr}$ \\
$lw_a$  & Parameter length-weight relationship & $\textrm{g cm}^{-1}$ \\
$lw_b$  & Parameter length-weight relationship & \_\\
$Q_\textrm{k}$  & Element-specific body content percentage of dry mass & \% \\
$f_\textrm{0}$  & Metabolic normalisation constant independent of body mass & $\textrm{g C} \textrm{g}^{-\alpha} \textrm{d}^{-1}$ \\
$\alpha$  & Mass-scaling exponent &  \_\\
$\theta$  & Activity scope & \_ \\
$v$  & Environmental temperature & \textdegree C \\
$h$  & trophic level & \_ \\
$r$  & Aspect ratio of caudal fin & \_ \\
$F_\textrm{0Nz}$  & Mass-specific turnover rate of N & $\textrm{g N} \textrm{g}^{-1} \textrm{d}^{-1}$ \\
$F_\textrm{0Pz}$  & Mass-specific turnover rate of P & $\textrm{g P} \textrm{g}^{-1} \textrm{d}^{-1}$ \\
$D_\textrm{k}$  & Elemental stoichiometry of diet & \% \\
\hline
\end{tabular}
\end{table}

\newpage

\hypertarget{growth-parameters}{%
\subsection{1. Growth parameters}\label{growth-parameters}}

\hypertarget{data-compilation}{%
\subsubsection{1.1 Data compilation}\label{data-compilation}}

We first compiled maximum lengths for all species with FishBase and used
these lengths for the \(l_{\infty}\) (REF). For \(\kappa\), we used a
standardized coefficient that describes the potential growth trajectory
of an individual if \(l_{\infty}\) were to be equal to equal to its
maximum length {[}i.e. \$k\_\{max\}\$, @Morais2019{]}. \(t0\) was kept
constant at 0 for all species.

We extracted the data for \(k_{max}\) from @Morais2019 and filtered out
only the species of our species list. As the Lenth-Frequency method
consistently overestimates kmax, we omitted the \(k_{max}\) estimates
coming from this method. In total, this selection process resulted in
439 observations of kmax for different species and temperatures.

Further, we used the otolith data provided by Morat et al., including
measurements of fishes from five Polynesian islands and fitted the Von
Bertalanffy growth models to all species at each location for which
there were at least 3 individuals. We fitted the model using Bayesian
regression models provided by fishgrowbot (REF).

After combining the two data sources, we obtained 496 estimates of
\(k_{max}\) for 181 species.

\hypertarget{data-analysis-and-extrapolation}{%
\subsubsection{1.2 Data analysis and
extrapolation}\label{data-analysis-and-extrapolation}}

\noindent Aside from phylogeny, \(k_{max}\) is mostly determined by body
size and temperature {[}@Morais2019{]}.

We applied a Bayesian hierarchical model to predict the growth rate of
fishes in function of body size, temperature and phylogeny:

\begin{equation}
\textrm{ln}kmax = (\beta_\textrm{0} + \gamma_\textrm{0phy}) +  \beta_\textrm{1} \textrm{ln}sizemax +  \beta_\textrm{2} sst + \epsilon,
    \label{kmax}
\end{equation}

\noindent where \(\textrm{ln}kmax\) represents the natural
log-transformed kmax value, \(\beta_\textrm{0}\) is the fixed-effect
intercept, \(\gamma_\textrm{0phy}\) is the vector of random-effect
coefficients that account for the resisual intercept variation, based on
the relatedness as described by the phylogeny, \(\beta_\textrm{1}\) is
the slope for the natural transformed maximum body size,
\(\beta_\textrm{2}\) is the slope for the average ambient sea surface
temperature, \(\epsilon\) is the residual variation. \noindent We used
uninformative priors and ran the model for 2000 iterations with a
warm-up of 1000 iteration for 4 chains. \noindent The model fit
confirmed a negative relationship of \(\textrm{ln}kmax\) with
\(\textrm{ln}size\), and a positive relationship with sea surface
temperature. The Bayesian R2 of the model was 0.73759 (95\%CI:
0.702-0.769). \noindent The phylogenetic heritability (equivalent to
Pagel'\(\lambda\)) was estimated as the proportion of total variance,
conditioned on the effects, attributable to the phylogeny(i.e.
\(\lambda = (\textrm{sd}(\gamma_\textrm{0phy})^2 / (\textrm{sd}(\gamma_\textrm{0phy})^2 + \epsilon^2)\)).
This calculation resulted in a phylogenetic signal of 0.74 (95\% CI:
0.70 - 0.77).

\noindent We extrapolated \(\kappa\) for all species across the full
temperature range in which those species occur in the database, with
temperature rounded to the °C, which results in 4712 unique temperature
and species combinations.\\
There is currently no streamlined method to make predictions to new
species from a phylogenetic regression model. We circumvent the issue by
extracting draws of the phylogenetic effect, \(\gamma_\textrm{0phy}\)
for each species included in the model. We subsequently predicted these
phylogenetic effects for missing species with the help of the function
phyEstimate in the picante package for R {[}@Kembel2010{]}. This
function uses phylogenetic ancestral state estimation to infer trait
values for new species on a phylogenetic tree by rerooting the tree to
the parent edge for the node to be predicted {[}@Kembel2012{]}. We
repeated this for all 100 trees and 1000 draws. Per draw, we averaged
the extrapolated values per species for the hundred trees. Then, by
combining the predicted phylogenetic effects with the global intercept
and slopes for body size and temperatures for each draw, we predicted
\(\kappa\) for each species. We only use one chain in order to keep
computational time reasonable. Finally, we summarised all \(\kappa\)
predictions per sst per species by taking the mean and standard
deviation across the 1000 draws.

\hypertarget{body-stoichiometry}{%
\subsection{2 Body stoichiometry}\label{body-stoichiometry}}

\hypertarget{data-collection}{%
\subsubsection{2.1 Data collection}\label{data-collection}}

\noindent 1633 individuals of 108 species and 25 families were collected
between 2015 and 2017 in Moorea, the Carribean, and Palmyra. Their gut
contents were removed, and the whole body was freeze-dried and ground to
powder with a Precellys homogeniser. \(Q_\textrm{k}\) (\%) were then
measured in the lab using standard methods. Ground samples were analysed
for \%C and \%N content using a CHN Carlo-Erba elemental analyzer
(NA1500) for \%P using dry oxidation-acid hydrolysis extraction followed
by a colorimetric analysis (Allen et al.~1974). Elemental content was
calculated based on dry mass.

\hypertarget{data-analysis-and-extrapolation-1}{%
\subsubsection{2.2 Data analysis and
extrapolation}\label{data-analysis-and-extrapolation-1}}

\noindent We fitted C, N and P contents (\%) through a hierarchical
phylogenetic multivariate normal model with phylogenetic effects and
random effects per species.

\begin{equation}
    \begin{bmatrix}Y_1 \\ Y_2 \\ Y_3\end{bmatrix} \sim MVNormal(\begin{bmatrix}mu1 \\ mu2 \\ mu3\end{bmatrix}, S),
    \label{cnp}
\end{equation}

\begin{equation}
    mu_{\textrm{n}\times\textrm{k}} = \beta_\textrm{0 k} + \gamma_{\textrm{0phy}\times\textrm{k}} + \gamma_{\textrm{0sp}\times\textrm{k}},
    \label{cnp2}
\end{equation}

\noindent where \(Y_1\), \(Y_2\) and \(Y_3\) are the \% content of
\(\textrm{C}\), \(\textrm{N}\), and \(\textrm{P}\) respectively,
\(mu_{\textrm{n}\times\textrm{k}}\) represents the average \% content of
element \(k\) (\(\textrm{C}\), \(\textrm{N}\), and \(\textrm{P}\)) per
species, \(\beta_\textrm{0 k}\) is the fixed-effect intercept for each
element \(k\), \(\gamma_{\textrm{0phy}\times\textrm{k}}\) is the matrix
of random-effect coefficients that account for the intercept variation,
based on the relatedness as described by the phylogeny per element k,
\(\gamma_{\textrm{0sp}\times\textrm{k}}\) is the matrix of random-effect
coefficients that account for the residual species-level intercept
variation per element k.

\noindent We used uninformative priors and ran the model for 2000
iterations with a warm-up of 1000 iteration for 4 chains. \noindent The
Bayesian R2 of the model was 0.39 (95\%CI: 0.36-0.42), 0.50 (95\%CI:
0.48-0.53), and 0.43 (95\%CI: 0.40-0.46) for C, N and P respectively.
\noindent The phylogenetic heritability was 0.41 (95\%CI: 0.28-0.55),
0.58 (95\%CI: 0.4-0.66), and 0.57 (95\%CI: 0.46-0.69) for C, N, and P
respectively.

\noindent As before, we used 1000 fitted draws for each species, and 100
phylogenetic trees to extrapolate to all species with unknown body
stoichiometry. Specifically, we used the phylopars function from the
Rphylopars package {[}@Bruggeman2009{]}. This function uses ancectral
state reconstruction and brownian motion, and takes the correlation
between C, N and P into account.

\hypertarget{diet-stoichiometry}{%
\subsection{3 Diet stoichiometry}\label{diet-stoichiometry}}

\hypertarget{data-collection-1}{%
\subsubsection{3.1 Data collection}\label{data-collection-1}}

\noindent We collected 571 adult individuals of 51 species between 2018
and 2019 in Mo'orea and Tetiaroa, and Mangareva, three Polynesian
islands. We extracted the stomach content and stored it in a 2ml tube.
After freezing the samples, we dry-froze all samples for at least 24
hours, and ground to powder. Then, samples were sent to the lab for
CNP\% content analysis using similar methods as for the fish body
stoichiometry.

\hypertarget{data-analysis-and-extrapolation-2}{%
\subsubsection{3.2 Data analysis and
extrapolation}\label{data-analysis-and-extrapolation-2}}

\noindent We used trophic guilds defined by Parravicini et al. (2020).
We fitted a multivariate Bayesian regression model to summarize CNP\%
content data per trophic guild with random effects at the species level.
This model had a median Bayesian R2 of 0.62, 0.62, and 0.48 for C, N and
P respectively.\\
Next, we extracted 1000 draws of the predicted the CNP\% per trophic
guild. Parravicini et al. (2020) provides the probability of reef fish
species to be assigned to each of the 8 defined trophic guilds. By
combining these probabilities with the predicted diet contents per
trophic guild, we finally estimated the diet CNP\% for each species in
our database. We then took the average and standard deviation across all
1000 draws.

\hypertarget{metabolic-parameters}{%
\subsubsection{4 Metabolic parameters}\label{metabolic-parameters}}

\hypertarget{data-collection-2}{%
\subsubsection{4.1 Data collection}\label{data-collection-2}}

In the period between 2018 and 2019, we collected 1393 individuals of 61
species and 18 families with a minimum of 3 replicates per species.
Individuals were collected using handnets and clove oil by scuba divers.

\hypertarget{metabolic-rate}{%
\subsubsection{4.2 metabolic rate}\label{metabolic-rate}}

To quantify standard metabolic rate (SMR) and maximum metabolic rate
(MMR), we conducted intermittent-closed respirometry experiments at 28°C
(Steffensen 1989; Clark, Sandblom, and Jutfelt 2013). After an
acclimatization and fasting period of 48 h in aquaria, the fish were
individually transferred to a water-filled tub at 28°C and manually
chased by the experimenter until exhausted (Norin and Malte 2011; Clark
et al.~2012). Then, they were placed in respirometry chambers submersed
in an ambient and temperature-controlled tank, where they were left for
\textasciitilde{}23 h. The intermittent respirometry cycles started
immediately after a fish was placed in its respirometry chamber. The
cycles consisted of a measurement (sealed) period followed by a flush
period during which the respirometry chambers were flushed with fully
aerated water from the ambient tank. Because fish were exhausted right
before entering the respirometry chambers, it is possible to measure the
approximate MMR. Depending on fish size, 8 respirometry chambers ranging
in volume (including tubes and pumps) from 0.4 to 4.4 L were run in
parallel, and measurement and flush periods lasted between 3 to 15 min
and 3 to 5 min, respectively. SMR was calculated as the average of the
10 \% lowest values measured during the entire period, after the removal
of outliers (Chabot, Steffensen, and Farrell 2016). MMR was calculated
from the slope of the first measurement period.

\hypertarget{data-analysis-and-extrapolation-3}{%
\subsubsection{4.3 Data analysis and
extrapolation}\label{data-analysis-and-extrapolation-3}}

To retrieve the parameters \(f_\textrm{0}\) (Metabolic normalisation
constant independent of body mass;
\(\textrm{g C} \textrm{g}^{-\alpha} \textrm{d}^{-1}\)) and \(\alpha\)
(mass-scaling exponent), and \(\theta\) (factorial activity scope), we
fitted a Bayesian mixed effect model predicting the log10-transformed
metabolic rate with the log10-transformed biomass including random
effects of family, species, and mr type (SMR or MMR) on both the
intercept and the species. We ran the model for 4000 iterations, with a
warm-up of 2000 iterations. Further, we used an informative prior for
the slope (\(\alpha \sim normal(0.8, 0.5)\). The model had a Bayesian R2
of 0.973 (95\%CI: 0.972-0.974). We then extracted the family-level
\(\alpha\) by summing the slope of the model with the effects of the
family on the slope of the SMR. We did this for 1000 iterations and then
took the mean and standard deviation. In a similar way we extracted the
family-level intercept for SMR, and then quantified mean and standard
deviation of \(f_\textrm{0}\) after the back-transformation of 1000
iterations of the intercept. Finally, \(\theta\) was quantified as
followed, based on the assumption that fishes rest 12h a day and they on
average spend the remaining 12 hours at a metabolic rate that is the
average of their SMR and MMR: \begin{equation}
    \theta = \frac{3\textrm{SMR} + \textrm{MMR}}{4\textrm{SMR}},
    \label{theta}
\end{equation} where 1000 iterations of the back-transformed
family-level intercepts were used for SMR and MMR. We then summarized
these predictions by taking the mean and standard deviation. We used the
family-level estimates for these three parameters for all species in our
database. For familues that were not represented in our respirometry
dataset, we used an average across all families.

\hypertarget{additional-parameters}{%
\subsection{5. Additional parameters}\label{additional-parameters}}

We retrieved the parameters \(lw_a\), \(lw_b\), \(h\), and \(r\) from
fishbase (REF). For the mass-specific turnover rates for N and
P(\(F_\textrm{0Nz}\); \(F_\textrm{0Pz}\)), we used the estimates
provided in (Schiettekatte et al.~2020).



\end{document}
